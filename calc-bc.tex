\documentclass[letterpaper]{report}
\usepackage{mathtools}
\usepackage{amsthm}
\usepackage{physics}

\usepackage[depth=3]{bookmark}
\setcounter{tocdepth}{3}

\usepackage{hyperref}
\hypersetup{
     colorlinks = true,
     linkcolor = black,
     }

\title{My Notes for AP Calculus BC}
\author{Luke Barlow}
\date{2023-2024}

\theoremstyle{definition}
\newtheorem{definition}{Definition}

%---------------------------------------------------
\begin{document}

\maketitle
\tableofcontents

%---------------------------------------------------
\chapter{Limits and Continuity}

\section{Computing Limits}

\section{Limits at Infinity}

\section{Continuity}

\section{Intermediate Value Theorem}

\section{Squeeze Theorem}


%---------------------------------------------------
\chapter{Differentiation and the Rate of Change}

\section{Tangent Lines and Rates of Change}

\section{The Derivative Function}

\section{Techniques of Differentiation}

\section{Product Rule and Quotient Rule}

\section{Derivatives of Trig Functions}

\section{The Chain Rule}


%---------------------------------------------------
\chapter{Topics in Differentiation}

\section{Implicit Differentiation}

\section{Derivatives of Logarithmic Functions}

\section{Derivatives of Exponential Functions}

\section{Derivatives of Inverse Functions}

\section{Related Rates}

\section{Local Linear Approximation}

\section{L'H\^{o}pital's Rule and Indeterminate Forms}


%---------------------------------------------------
\chapter{The Derivative in Graphing and Applications}

\section{Increase, Decrease, and Concavity}

\section{Relative Extrema}

\section{Absolute Maxima and Minima}

\section{Applied Max and Min Problems}

\section{Rectilinear Motion}

\section{Mean Value Theorem}


%---------------------------------------------------
\chapter{Integration}

\section{Overview of Area}

\section{The Indefinite Integral}

\section{Slope Fields}

\section{Integration By Substitution}

\section{Area as a Limit and Riemann Sums}

\section{Exact Area Under a Curve (Trapezoid Rule)}

\section{The Definite Integral}

\section{The Accumulation Function}

\section{The Fundamental Theorem of Calculus}

\section{Total Change Theorem}

\section{Average Value}

\section{Definite Integrals by Substitution}


%---------------------------------------------------
\chapter{Applications of the Definite Integral}

\section{Area Between Two Curves}

\section{Volumes by Slicing}

\section{Disks and Washers}

\section{Length of a Plane Curve}


%---------------------------------------------------
\chapter{Principles of Integral Evaluation}

\section{Integration by Parts}

\section{Integration of Rational Functions by Partial Fractions}

\section{Improper Integrals}


%---------------------------------------------------
\chapter{Differential Equations}

\section{Logistic Growth}

\section{Separable Equations}

\section{Exponential Growth and Decay}

\section{Euler's Method}


%---------------------------------------------------
\chapter{Infinite Series}

\section{Defining Convergent and Divergent Infinite Series}

\section{Convergence Tests}

\subsection{Geometric Series Test}
\begin{definition}
A series in the form $\sum{ar^n} = a + ar + ar^2 + ar^3 + ... + ar^n ...$
is called a geometric series with ratio $r$.
\end{definition}
An infinite geometric series with ratio $r$ diverges if $|r|\geq1$.
If $|r|<1$, we can say that the series converges by the 
\textbf{geometric series test}. The infinite sum of this series is
\[ \sum_{n=0}^\infty ar^n = \frac{a}{1-r} \]

\subsection{nth Term Test}

\subsection{Integral Test}
\begin{definition}
If $f$ is positive, continuous, and decreasing for $x\geq{}m\geq{}1$ where
m is a positive integer and $a_n=f(x)$, then $\sum_{n=1}^\infty a_n$ and
$\int_1^\infty f(x) \, \mathrm{d}x$ either both converge or diverge.
\end{definition}
Use implicit integration to determine whether the integral converges or diverges.
\textbf{Note:} The answer to the limit or the integral is \textit{not} the 
sum of the infinite series.

\subsection{p-series and Harmonic Series}
\begin{definition}
A p-series is an infinite series in the form
\[ \sum_{n=1}^\infty \frac{1}{n^p}=\frac{1}{1^p}+\frac{1}{2^p}
    +\frac{1}{3^p}+...+\frac{1}{n^p}+... \]
where p is a positive number.
\end{definition}
The p-series will converge if $p>1$ and diverge if $1<p\leq1$

\subsection{Comparison Tests}

\subsubsection{Direct Comparison Test}

\subsubsection{Limit Comparison Test}

\subsection{Polynomial Test}

\subsection{Alternating Series}

\subsection{Ratio Test}
A series $\sum{a_n}$ is absolutely convergent if the limit of the ratio
of successive terms,
\[ \lim_{n \rightarrow \infty}\frac{|a_{n+1}|}{|a_n|} \]
is less than 1. If the limit is greater than 1 or approaches $\infty$, then
the series diverges. The ratio test is inconclusive if the limit equals 1.


\subsection{Root Test}

\section{Absolute and Conditional Convergence}

\section{Power Series}

\section{Error Bounds}

\section{Taylor Series}

%---------------------------------------------------
\chapter{Parametric, Polar, and Vector-Valued Functions}

\section{Parametric Equations}
Parametric equations are functions of a single, independent variable 
(usually $t$) called a parameter.
Parametric equations represent the coordinates that make up a parametric curve
in the form $(x(t), y(t))$.

\subsection{Derivatives}
\begin{definition}[First Derviative]
    For a smooth curve $C$ represented by $x = x(t)$ and $y = y(t)$,
    the slope of the line tangent to $C$ at $(x,y)$ is
\[ \dv{y}{x} = \frac{\dv{y}{t}}{\dv{x}{t}} \]
as long as $\dv{x}{t} \neq 0$. 
\end{definition}

\subsection{Arc Length}

\section{Vector-Valued Functions}

\section{Polar Functions}

\end{document}
